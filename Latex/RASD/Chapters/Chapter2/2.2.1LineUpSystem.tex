\subsection{Line-Up}

This part of the whole system manages the action of making a Line-Up Request for a specific supermarket. The Customer uses the mobile phone interface to make a Line-up Request indicating the involved supermarket and the mean of transport he intends to use to reach it. Then, the system add the Customer in the Real-Time Queue of that store and shows him his Waiting Time. In addition, the system sends to the Customer a QR code that will be scanned both at the entrance and at the exit to monitor the affluence and to update the Real-Time Queue. For preventing Customers to arrive both too early or too late, the QR Codes have a validity which spans in an interval of time of 5 minutes with respect to the entrance time set by the system. \\
Finally, the system calculates the time needed by the Customer to reach the supermarket and sends him a notification when he has to start.
It is important to underline that the system allows the Customer to make a single Line-Up Request at a time to permit a more fair distribution of the timeslots available. Also, for usage semplicity reasons, it permits only to cancel and not to modify the Request. Moreover, if the Line-Up Request is sent when the supermaket is closed or if the entrance time foreseen by the Waiting Time is after the closing time of the supermarket, the Request will be postponed to the first available slot. 