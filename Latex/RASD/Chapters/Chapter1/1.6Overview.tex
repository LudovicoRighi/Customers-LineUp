\section{Document's Overview}

The RASD document is structured by five chapters as describes below:\\

\begin{itemize}
	\item Chapter 1: it represent an introduction. It is formed by a general description of the purpose of the application and a list of the goals which the system has to achieve. In addition, it presents a detailed description of the software's functionalities, specifying, in particular, the application domain through the “World and Machine phenomena” criterion.
	\item Chapter 2: it contains an overall description of the project such as a detailed characterization of the software's main functions. It presents also the general factors that can affect the system's behavior, such as user characteristics and project constraints.
	\item Chapter 3: it represent the body of the document. It contains the main requirements of the software, and the mockups related to the two different interfaces of the system. It lists some scenarios to show how the system acts in real world situations. They are followed by all the requirements necessary to reach the given goals and linked with the related domain assumptions. Lastly, the non-functional requirements are defined through performance requirements, design constraints and
	software system attributes.
	\item Chapter 4: it contains the Alloy model of some important aspects, with all the related comments and documentation in order to show in a formal way how the project has been structured.
	\item Chapter 5: it contains the effort, in terms of hours, which each member of the group spent working on the project.
	\item Chapter 6: it contains the list of any other documents or Web addresses to which this RASD refers.
\end{itemize}