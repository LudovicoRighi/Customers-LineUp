\chapter{Testing}

This chapter present the approach with which testing has been carried out, starting with the planning and the approach that has been adopted in section 5.1 and providing also, in section 5.2 and 5.3, the Unit testing  and the Integration test cases.
 

\section{Testing Plan}
As already stated in the Design Document, the approach that has been choice for performing the testing is Bottom-Up: the reasons why this choice has been taken have been described in the DD, but certainly the most important one is the simplicity of the procedure, which typically allows to reach better results in a more efficient way. Moreover, since also the Integration strategy follows a Bottom-Up approach, it is possible to perform testing and integration in parallel with the implementation in a quite straightforward way: in fact, after having implemented a component, Unit testing is perfomed directly by programmers and as the system grows through the Integration testing it is checked that the different modules interacts correclty. At the end of this incremental process, once the whole system is complete, and accurate System testing will ensure that every requirement has been satisfied.

\section{Unit Testing}
The Unit testing has been carried out by means of the Junit framework, which is useful for automatizing them; in this way, if they have to be repeated in the future after the introduction of new extentions, the procedure of testing will be very time and effort effective. The test cases are implemented directly in the source code in the following classes, which can be found at CLup\textbackslash CLupEJB\textbackslash ejbModule\textbackslash it\textbackslash clup\textbackslash test: CustomerTest.java, LineUpRequest.java, Supermarket.java, TestRunner.java. \\
 Notice that, in order to reproduce them, it is necessary to have Junit installed.

\section{Integration Testing}
In this section are indicated the main Test Cases that have been used in order to perform the Integration testing together with their outcomes. This testing has been performed manually since, for the moment, the developed software is very simple and can be checked immediately. It is however evident that when new functionalities will be introduced, such as the possibility for Customer of doing Bookings and the optimization of the supermarkets based on the departments' affluence, an automatized testing technique will be needed.



\subsection{Test Case 1: Customer Registration}
The correct URL string that we have to pass is formatted like: \\
{\footnotesize http:localhost:8080/CLupWEB/signup?username=myUsername\&email=myEmail\&password=myPassword } \\
For brevity in the table the three parameters are refered to as U, E and P.
\begin{center}
    \begin{tabular}{ |  p{8cm} | p{8cm} |}
    \hline
     \textbf{Test ID}  & TC1 \\ \hline
     \textbf{Components}  & CustomerService, Customer\\ \hline
    \textbf{Input} & \textbf{Output}  \\ \hline
    Missing one or more param between U, E and P & Failed: username, password, email are all mandatory   \\ \hline
    U=(usernameAlreadyInDB) & Failed: the username must be unique \\ \hline
    E=(emailAlreadyInDB) & Failed: the email must be unique \\ \hline
    U="Marco", E="marco@gmail.com", P="gatto"  & Success 	   \\ \hline
    \end{tabular}
\end{center}



\subsection{Test Case 2: Customer Login}
The correct URL string that we have to pass is formatted like: \\
http:localhost:8080/CLupWEB/login?username=myUsername\&password=myPassword \\
For brevity in the table the two parameters are refered to as U and P.
\begin{center}
    \begin{tabular}{ |  p{8cm} | p{8cm} |}
    \hline
     \textbf{Test ID}  & TC2 \\ \hline
     \textbf{Components}  & CustomerService, Customer\\ \hline
    \textbf{Input} & \textbf{Output}  \\ \hline
    Missing one or more param between U and P & Failed: username and password are both mandatory   \\ \hline
    U=(invalidUsername) & Failed: the username does not exist \\ \hline
    U=(validUsername), P=(wrongPassword) & Failed: the password for this username is wrong \\ \hline
    U=(validUsername), P=(correctPassword) & Success 	   \\ \hline
    \end{tabular}
\end{center}



\subsection{Test Case 3: Add LineUpRequest}
The correct URL string that we have to pass is formatted like: \\
{\footnotesize http:localhost:8080/CLupWEB/addlineupreq?username=myUsername\&marketId=myMarketId} \\
For brevity in the table the two parameters are refered to as U and M.
\begin{center}
    \begin{tabular}{ |  p{8cm} | p{8cm} |}
    \hline
     \textbf{Test ID}  & TC3 \\ \hline
     \textbf{Components}  & CustomerService, Customer, Supermarket\\ \hline
    \textbf{Input} & \textbf{Output}  \\ \hline
    Missing one or more param between U and M & Failed: username and marketId are both mandatory   \\ \hline
    U=(invalidUsername) & Failed: the username does not exist \\ \hline	
    M=(invalidMarketId) & Failed: the marketId does not exist \\ \hline
    U=(userHasAlreadyLinedUp), M=(validMarket) & Failed: the username cannot have more than one LineUpRequest at a time\\ \hline	
    U=(username), M=(validMarket) & Success\\ \hline	
    \end{tabular}
\end{center}




\subsection{Test Case 4: Delete LineUpRequest}
The correct URL string that we have to pass is formatted like: \\
{\footnotesize http:localhost:8080/CLupWEB/dellineupreq?requestId=myRequestId} \\
For brevity in the table the parameter is refered to as R.
\begin{center}
    \begin{tabular}{ |  p{8cm} | p{8cm} |}
    \hline
     \textbf{Test ID}  & TC4 \\ \hline
     \textbf{Components}  & CustomerService\\ \hline
    \textbf{Input} & \textbf{Output}  \\ \hline
    Missing param R & Failed: requestId is mandatory   \\ \hline
    R=(invalidRequestId) & Failed: the requestId does not exist \\ \hline	
    R=(validRequestId) & Success \\ \hline	
    \end{tabular}
\end{center}


\subsection{Test Case 5: Customer Entrance}
The correct URL string that we have to pass is formatted like: \\
{\footnotesize http:localhost:8080/CLupWEB/entrlineup?requestId=myRequestId} \\
For brevity in the table the parameter is refered to as R.
\begin{center}
    \begin{tabular}{ |  p{8cm} | p{8cm} |}
    \hline
     \textbf{Test ID}  & TC5 \\ \hline
     \textbf{Components}  & CustomerService\\ \hline
    \textbf{Input} & \textbf{Output}  \\ \hline
   Missing param R & Failed: requestId is mandatory   \\ \hline
    R=(invalidRequestId) & Failed: the requestId does not exist \\ \hline	
    R=(validRequestId) & Success \\ \hline
    \end{tabular}
\end{center}
\subsection{Test Case 6: Retrieve Supermarkets}
The correct URL string that we have to pass is formatted like: \\
{\footnotesize http:localhost:8080/CLupWEB/retrmarkets}
There are no parameters to be passed.
\begin{center}
    \begin{tabular}{ |  p{8cm} | p{8cm} |}
    \hline
     \textbf{Test ID}  & TC1.0 \\ \hline
     \textbf{Components}  & CustomerService\\ \hline
    \textbf{Input} & \textbf{Output}  \\ \hline
    (correct URL string) & Success \\ \hline
    \end{tabular}
\end{center}







