\section{Other Design Decisions}

This section describes decision about algorithms and data structures used in Customer Line-up. In particular to compute the Waiting Time (see 2.7.1) and the Booking optimization (see 2.7.2).


\subsection{Next Waiting Time}

For this algorithm, we need two data structures.
\begin{itemize}
	\item A List that contains all the Time To Live (TTL) of all the people inside the store.
	\item A List that contains all the customer that are queuing from their home.
\end{itemize}
A Key parameter is AIT (Average Inside Time): the average time that people who have lined-up or are entered with the ticket stay in the supermarket.\\
The AIT is calculated:
\begin{itemize}
	\item For No-Tech Customers as the weighted average over days and times of the grocery shopping times of a specific supermarket.
	\item For Customers who have Lined-Up as as the weighted average over days and times of their grocery shopping times.
	\item For Customers who have Booked it was previouly indicated as Expected Duration or, only for Long Term Customers, can be infered by the System from their past visits.
\end{itemize}
The TTL is calculated:
\begin{itemize}
	\item For No-Tech Customers and Customers who have lined-up as the difference from their AIT and the time they already spent inside the store.
	\item For Customers who have Booked as the difference from the Expected Duration and the time they already spent inside the store.
\end{itemize}
If the supermarket is not full, there are not people who are queuing and the waiting time is zero. When the supermarket is full, this algorithm is used to infer the waiting time.\\
\lstinputlisting[language=java, basicstyle=\footnotesize]{WaitingTime.java}

\subsection{Booking Optimization}

Customer Line-up permits, on average, to use only the 95\% of the total capacity of every supermarket provided by the Prime Minister's Decree (DPCM). This choice has been taken to be more careful and to prevent emergencies. This value of 95\% can can increase by 5\% depending on the value of the AADs (average affluence of a department).\\
To compute this value we need two data structures:
\begin{itemize}
	\item A list for every Booked customer that contains the probability to go to a specific department. For example, if a customer indicates items that belongs to 4 different departments, the probability to reach them is 0.25 and the others is 0.
	\item A list that contains the preview data structure for all the customers who have booked in a specific time slot.
\end{itemize}
For every department is computed its AAD. The ADD is the sum of all the probabilities of every booking customer to go to the specific department divided by the capacity of that one.\\
If an AAD is higher than 1 for at least one department, then the maximum number of people inside the store remains the 95\% of the supermarket capacity. Else it is the 95\% + (5\% * (number of booked people / capacity)).\\
To better dispay the AAD formula, the second data structure is represented as a matrix where the columns (j) are the departments and the  rows (i) are the customers. \[AAD_{j} = \sum_i{M_{ij}} / DepCapacity_{j}\]

